\documentclass[12pt]{article}
\usepackage[legalpaper, portrait, margin=0.75in]{geometry}
\usepackage{amsmath}
\usepackage{amssymb}
\usepackage{amsthm}
\usepackage{physics}
\usepackage{mathtools}
\usepackage{braket}
\usepackage{tikz}
\usepackage{quantikz}
\usepackage{multicol}
\usepackage{xcolor}
\usepackage{fancyhdr}
\usepackage{titlesec}
\usepackage{multirow}
\usepackage{array}
\usepackage{booktabs}
\usepackage{quantikz}
\usepackage{tabularx}


% Page setup
\pagestyle{fancy}
\fancyhf{}
\fancyhead[L]{Quantum Computing LaTeX Quick Reference}
\fancyhead[R]{\thepage}
\setlength{\headheight}{14.5pt}

% Title formatting
\titleformat{\section}{\large\bfseries\color{blue!70!black}}{\thesection}{1em}{}
\titleformat{\subsection}{\normalsize\bfseries\color{blue!50!black}}{\thesubsection}{1em}{}

\title{\textbf{\Large Quantum Computing LaTeX Quick Reference}}
\author{Quick Command Sheet for Quantum Computing Lecture Notes}
\date{}

% =============================================================================
% QUANTUM STATE NOTATION
% =============================================================================

% Basic quantum states
% \newcommand{\ket}{|#1\rangle}
% \newcommand{\bra}{\langle#1|}
% \newcommand{\braket}{\langle#1|#2\rangle}
% \newcommand{\ketbra}{|#1\rangle\langle#2|}
\newcommand{\expectation}[1]{\langle#1\rangle}

\newcommand{\ketphi}{\ket{\phi}}
\newcommand{\braphi}{\bra{\phi}}
\newcommand{\ketpsi}{\ket{\psi}}
\newcommand{\brapsi}{\bra{\psi]}}

% Common quantum states
\newcommand{\qzero}{\ket{0}}
\newcommand{\qone}{\ket{1}}
\newcommand{\qplus}{\ket{+}}
\newcommand{\qminus}{\ket{-}}
\newcommand{\qpsi}{\ket{\psi}}
\newcommand{\qphi}{\ket{\phi}}
\newcommand{\qchi}{\ket{\chi}}

% Bell states
\newcommand{\bellphi}{\ket{\Phi^+}}
\newcommand{\bellphim}{\ket{\Phi^-}}
\newcommand{\bellpsi}{\ket{\Psi^+}}
\newcommand{\bellpsim}{\ket{\Psi^-}}

% GHZ and W states
\newcommand{\GHZ}{\ket{\text{GHZ}}}
\newcommand{\Wstate}{\ket{\text{W}}}

% =============================================================================
% QUANTUM OPERATORS AND GATES
% =============================================================================

% Pauli matrices
\newcommand{\sx}{\sigma_x}
\newcommand{\sy}{\sigma_y}
\newcommand{\sz}{\sigma_z}
\newcommand{\si}{\sigma_0}
\newcommand{\pauliX}{\sigma_X}
\newcommand{\pauliY}{\sigma_Y}
\newcommand{\pauliZ}{\sigma_Z}

% Common quantum gates
\newcommand{\hadamard}{H}
\newcommand{\cnot}{\text{CNOT}}
\newcommand{\ccnot}{\text{CCNOT}}
\newcommand{\toffoli}{\text{Toffoli}}
\newcommand{\fredkin}{\text{Fredkin}}
% \newcommand{\swap}{\text{SWAP}}

% Rotation gates
\newcommand{\rx}[1]{R_x(#1)}
\newcommand{\ry}[1]{R_y(#1)}
\newcommand{\rz}[1]{R_z(#1)}
\newcommand{\rn}[2]{R_{\hat{n}}(#1, #2)}

% Phase gates
\newcommand{\phaseS}{S}
\newcommand{\phaseT}{T}
\newcommand{\phaseP}[1]{P(#1)}

% Measurement
% \newcommand{\measure}{\mathcal{M}}
% \newcommand{\proj}[1]{P_{#1}}
\newcommand{\projzero}{\proj{0}}
\newcommand{\projone}{\proj{1}}

% =============================================================================
% QUANTUM OPERATIONS AND PROCESSES
% =============================================================================

% Unitary operations
\newcommand{\unitary}{\mathcal{U}}
\newcommand{\unitarydagger}{\mathcal{U}^\dagger}

% Quantum channels
\newcommand{\channel}{\mathcal{E}}
\newcommand{\noise}{\mathcal{N}}
\newcommand{\depolar}{\mathcal{D}}
\newcommand{\bitflip}{\mathcal{B}}
\newcommand{\phaseflip}{\mathcal{P}}

% Tensor products
\newcommand{\tensor}{\otimes}
% \newcommand{\bigotimes}{\bigotimes}

% Trace operations
% \newcommand{\tr}{\text{Tr}}
\newcommand{\ptr}[1]{\text{Tr}_{#1}}

% =============================================================================
% QUANTUM INFORMATION MEASURES
% =============================================================================

% Entropies
\newcommand{\entropy}[1]{H(#1)}
\newcommand{\condentropy}[2]{H(#1|#2)}
\newcommand{\mutualinfo}[2]{I(#1:#2)}
\newcommand{\vonneumann}[1]{S(#1)}

% Distances and fidelities
\newcommand{\fidelity}[2]{F(#1, #2)}
\newcommand{\tracedist}[2]{D(#1, #2)}

% =============================================================================
% QUANTUM ALGORITHMS
% =============================================================================

% Algorithm names
\newcommand{\QFT}{\text{QFT}}
\newcommand{\iQFT}{\text{QFT}^{-1}}
\newcommand{\grover}{\text{Grover}}
\newcommand{\shor}{\text{Shor}}

% Oracle notation
\newcommand{\oracle}{O}
\newcommand{\oraclef}{O_f}
\newcommand{\oracleg}{O_g}

% Amplitudes
\newcommand{\amp}[1]{\alpha_{#1}}
\newcommand{\betamp}[1]{\beta_{#1}}

% =============================================================================
% QUANTUM ERROR CORRECTION
% =============================================================================

% Stabilizers
\newcommand{\stab}{\mathcal{S}}
\newcommand{\stabgen}{g}
\newcommand{\syndrome}{s}

% Error operators
\newcommand{\error}{E}
\newcommand{\errorX}{X}
\newcommand{\errorY}{Y}
\newcommand{\errorZ}{Z}

% Code spaces
\newcommand{\codespace}{\mathcal{C}}
\newcommand{\codeword}{\ket{\overline{\psi}}}

% =============================================================================
% QUANTUM COMPLEXITY
% =============================================================================

% Complexity classes
\newcommand{\BQP}{\text{BQP}}
\newcommand{\QMA}{\text{QMA}}
\newcommand{\QPSPACE}{\text{QPSPACE}}

% =============================================================================
% QUANTUM CIRCUITS AND DIAGRAMS
% =============================================================================

% Circuit elements (for use with quantikz)
% \newcommand{\qw}{\qw}
% \newcommand{\gate}[1]{\gate{#1}}
% \newcommand{\ctrl}{\ctrl{1}}
% \newcommand{\targ}{\targ{}}
% \newcommand{\meter}{\meter{}}

% =============================================================================
% MATHEMATICAL UTILITIES
% =============================================================================

% Complex numbers
\newcommand{\ii}{\mathrm{i}}
% \newcommand{\real}{\text{Re}}
\newcommand{\imag}{\text{Im}}
\newcommand{\conj}[1]{#1^*}

% Matrix operations
% \newcommand{\rank}{\text{rank}}
\newcommand{\nullspace}{\text{null}}
\newcommand{\colspace}{\text{col}}
\newcommand{\rowspace}{\text{row}}

% Probability
\newcommand{\prob}[1]{\text{Pr}[#1]}
\newcommand{\expect}[1]{\mathbb{E}[#1]}
% \newcommand{\var}[1]{\text{Var}[#1]}

% Sets
\newcommand{\hilbert}{\mathcal{H}}
\newcommand{\density}{\mathcal{D}}
\newcommand{\positive}{\mathcal{P}}

% =============================================================================
% QUANTUM PHYSICS NOTATION
% =============================================================================

% Time evolution
\newcommand{\hamiltonian}{\mathcal{H}}
\newcommand{\timeevol}[1]{e^{-\ii #1 t/\hbar}}
% \newcommand{\hbar}{\hbar}

% Angular momentum
\newcommand{\angmom}{\mathbf{J}}
\newcommand{\spin}{\mathbf{S}}
\newcommand{\orbital}{\mathbf{L}}

% Commutators and anticommutators
% \newcommand{\comm}[2]{[#1, #2]}
\newcommand{\anticomm}[2]{\{#1, #2\}}

% =============================================================================
% ENVIRONMENTS FOR QUANTUM COMPUTING
% =============================================================================

% Theorem-like environments (already defined in main template)
% Additional quantum-specific environments
\theoremstyle{definition}
\newtheorem{algorithm}{Algorithm}[section]
\newtheorem{protocol}{Protocol}[section]
\newtheorem{circuit}{Circuit}[section]

\theoremstyle{remark}
\newtheorem{quantum-note}{Quantum Note}[section]
\newtheorem{classical-note}{Classical Note}[section]

\begin{document}
\maketitle
\thispagestyle{fancy}

\begin{multicols}{2}

\section{Basic Quantum State Notation}

\subsection{State Vectors}
\begin{alignat}{2}
\text{Ket:}          &\quad && \ket{\psi} \text{ or } \qpsi \\
\text{Bra:}          &\quad && \bra{\psi} \\
\text{Braket:}       &\quad && \braket{\psi | \phi} \\
\text{Outer product:}&\quad && \ketbra{\psi}{\phi} \\
\text{Expectation:}  &\quad && \expectation{A} = \braket{\psi}{A|\psi}
\end{alignat}

\subsection{Common States}
\begin{alignat}{2}
\text{Computational basis:} &\quad && \qzero, \qone \\
\text{Superposition:}       &\quad && \qplus = \frac{1}{\sqrt{2}}(\qzero + \qone) \\
                             &\quad && \qminus = \frac{1}{\sqrt{2}}(\qzero - \qone) \\
\text{Bell states:}         &\quad && \bellphi, \bellphim, \bellpsi, \bellpsim \\
\text{Multi-qubit:}         &\quad && \GHZ, \Wstate
\end{alignat}


\section{Quantum Gates and Operations}

\subsection{Single-Qubit Gates}
\begin{alignat}{2}
\text{Pauli matrices:} &\quad && \pauliX, \pauliY, \pauliZ \\
\text{Hadamard:}       &\quad && \hadamard \\
\text{Phase gates:}    &\quad && \phaseS, \phaseT, \phaseP{\theta} \\
\text{Rotations:}      &\quad && \rx{\theta}, \ry{\theta}, \rz{\theta}
\end{alignat}

\subsection{Multi-Qubit Gates}
\begin{alignat}{2}
\text{CNOT:}     &\quad && \cnot \\
\text{Toffoli:}  &\quad && \toffoli \text{ or } \ccnot \\
\text{Fredkin:}  &\quad && \fredkin \\
\text{SWAP:}     &\quad && \text{SAWP}
\end{alignat}

\section{Quantum Measurements}

\begin{alignat}{2}
\text{Measurement operator:} &\quad && \mathcal{M} \\
\text{Projectors:}           &\quad && \projzero, \projone, \P_{i} \\
\text{Probability:}          &\quad && \prob{i} = \braket{\psi}{\proj{i}|\psi}
\end{alignat}

% \newcommand{\measure}{\mathcal{M}}
% \newcommand{\proj}[1]{P_{#1}}


\section{Quantum Channels and Noise}

\begin{alignat}{2}
\text{General channel:} &\quad && \channel(\rho) \\
\text{Depolarizing:}    &\quad && \depolar_p(\rho) \\
\text{Bit flip:}        &\quad && \bitflip_p(\rho) \\
\text{Phase flip:}      &\quad && \phaseflip_p(\rho)
\end{alignat}


\section{Quantum Information Measures}

\begin{alignat}{2}
\text{Von Neumann entropy:} &\quad && \vonneumann{\rho} \\
\text{Conditional entropy:} &\quad && \condentropy{A}{B} \\
\text{Mutual information:}  &\quad && \mutualinfo{A}{B} \\
\text{Fidelity:}             &\quad && \fidelity{\rho}{\sigma} \\
\text{Trace distance:}       &\quad && \tracedist{\rho}{\sigma}
\end{alignat}


\section{Quantum Algorithms}

\subsection{Fourier Transform}
\begin{alignat}{2}
\text{QFT:}         &\quad && \QFT\ket{x} = \frac{1}{\sqrt{N}} \sum_{k=0}^{N-1} e^{2\pi i xk/N}\ket{k} \\
\text{Inverse QFT:} &\quad && \iQFT
\end{alignat}

\subsection{Oracles}
\begin{alignat}{2}
\text{Boolean oracle:} &\quad && \oraclef\ket{x}\ket{y} = \ket{x}\ket{y \oplus f(x)} \\
\text{Phase oracle:}   &\quad && \oraclef\ket{x} = (-1)^{f(x)}\ket{x}
\end{alignat}

\section{Error Correction}

\begin{alignat}{2}
\text{Stabilizer group:}  &\quad && \stab = \langle g_1, g_2, \ldots, g_k \rangle \\
\text{Syndrome:}          &\quad && \syndrome = (s_1, s_2, \ldots, s_k) \\
\text{Code space:}        &\quad && \codespace \\
\text{Logical codeword:}  &\quad && \codeword
\end{alignat}


\section{Mathematical Utilities}

\subsection{Complex Numbers}
\begin{alignat}{2}
\text{Imaginary unit:}     &\quad && \ii \\
\text{Real part:}          &\quad && \real(z) \\
\text{Imaginary part:}     &\quad && \imag(z) \\
\text{Complex conjugate:}  &\quad && \conj{z}
\end{alignat}

\subsection{Linear Algebra}
\begin{alignat}{2}
\text{Tensor product:}  &\quad && A \tensor B \\
\text{Trace:}           &\quad && \tr(A) \\
\text{Partial trace:}   &\quad && \ptr{B}(\rho_{AB}) \\
\text{Rank:}            &\quad && \rank(A)
\end{alignat}

\subsection{Probability}
\begin{alignat}{2}
\text{Probability:}  &\quad && \prob{\text{event}} \\
\text{Expectation:}  &\quad && \expect{X} \\
\text{Variance:}     &\quad && \var{X}
\end{alignat}

\end{multicols}


\section{Circuit Notation (with quantikz)}

Basic circuit elements:
\begin{verbatim}
\begin{quantikz}
\lstick{$\ket{0}$} & \gate{H} & \ctrl{1} & \meter{} \\
\lstick{$\ket{0}$} & \qw & \targ{} & \qw
\end{quantikz}
\end{verbatim}

\section{Complexity Classes}

\begin{alignat}{2}
\text{Bounded-error quantum polynomial:} &\quad && \BQP \\
\text{Quantum Merlin-Arthur:}           &\quad && \QMA \\
\text{Quantum polynomial space:}        &\quad && \QPSPACE
\end{alignat}


\section{Usage Examples}

\subsection{Quantum Teleportation Protocol}
The quantum teleportation protocol transfers the state $\qpsi = \alpha\qzero + \beta\qone$ using the following steps:

\begin{protocol}
\begin{enumerate}
\item Alice and Bob share the Bell state $\bellphi = \frac{1}{\sqrt{2}}(\ket{00} + \ket{11})$
\item Alice performs a Bell measurement on her qubit and the qubit to be teleported
\item Alice sends the classical result to Bob
\item Bob applies the appropriate correction based on Alice's measurement
\end{enumerate}
\end{protocol}

\subsection{Grover's Algorithm}
Grover's algorithm amplifies the amplitude of marked states:

\begin{algorithm}
\begin{enumerate}
\item Initialize: $\ket{\psi_0} = \frac{1}{\sqrt{N}}\sum_{x=0}^{N-1}\ket{x}$
\item Apply $G = -\hadamard^{\tensor n}\proj{0}^{\tensor n}\hadamard^{\tensor n}\oraclef$ for $O(\sqrt{N})$ iterations
\item Measure to obtain the marked state with high probability
\end{enumerate}
\end{algorithm}

\subsection{Quantum Error Correction Example}
The 3-qubit bit flip code protects against single bit flip errors:

\begin{circuit}
Encoding: $\ket{0} \rightarrow \ket{000}$, $\ket{1} \rightarrow \ket{111}$

Syndrome measurement: $s_1 = Z_1Z_2$, $s_2 = Z_2Z_3$
\end{circuit}

\section{Common Quantum Gates}

\begin{table}[h!]
\centering
\renewcommand{\arraystretch}{1.6}
\begin{tabularx}{\textwidth}{X X}
% --- Left column: Single-Qubit ---
\begin{tabular}{cl}
\toprule
\textbf{Gate} & \textbf{Matrix} \\
\midrule
Hadamard ($H$) &
$\dfrac{1}{\sqrt{2}} \begin{bmatrix}
1 & 1 \\
1 & -1
\end{bmatrix}$ \\
Identity ($I$) &
$\begin{bmatrix}
1 & 0 \\
0 & 1
\end{bmatrix}$ \\
Phase ($S$) &
$\begin{bmatrix}
1 & 0 \\
0 & i
\end{bmatrix}$ \\
T-gate ($T$) &
$\begin{bmatrix}
1 & 0 \\
0 & e^{i\pi/4}
\end{bmatrix}$ \\
NOT ($X$) &
$\begin{bmatrix}
0 & 1 \\
1 & 0
\end{bmatrix}$ \\
Y-gate ($Y$) &
$\begin{bmatrix}
0 & -i \\
i & 0
\end{bmatrix}$ \\
Z-gate ($Z$) &
$\begin{bmatrix}
1 & 0 \\
0 & -1
\end{bmatrix}$ \\
Rotation $R(\theta)$ / Phase $P(\theta)$ &
$\begin{bmatrix}
1 & 0 \\
0 & e^{i\theta}
\end{bmatrix}$ \\
\bottomrule
\end{tabular}
&
% --- Right column: Multi-Qubit ---
\begin{tabular}{cl}
\toprule
\textbf{Gate} & \textbf{Matrix} \\
\midrule
CNOT ($CX$) &
$\begin{bmatrix}
1 & 0 & 0 & 0 \\
0 & 1 & 0 & 0 \\
0 & 0 & 0 & 1 \\
0 & 0 & 1 & 0
\end{bmatrix}$ \\
Controlled-Z ($CZ$) &
$\begin{bmatrix}
1 & 0 & 0 & 0 \\
0 & 1 & 0 & 0 \\
0 & 0 & 1 & 0 \\
0 & 0 & 0 & -1
\end{bmatrix}$ \\
Controlled-$U$ ($CU$) &
$\begin{bmatrix}
1 & 0 & 0 & 0 \\
0 & 1 & 0 & 0 \\
0 & 0 & U_{00} & U_{01} \\
0 & 0 & U_{10} & U_{11}
\end{bmatrix}$ \\
SWAP &
$\begin{bmatrix}
1 & 0 & 0 & 0 \\
0 & 0 & 1 & 0 \\
0 & 1 & 0 & 0 \\
0 & 0 & 0 & 1
\end{bmatrix}$ \\
Toffoli (CCNOT) &
\scriptsize
$\begin{bmatrix}
1 & 0 & 0 & 0 & 0 & 0 & 0 & 0 \\
0 & 1 & 0 & 0 & 0 & 0 & 0 & 0 \\
0 & 0 & 1 & 0 & 0 & 0 & 0 & 0 \\
0 & 0 & 0 & 1 & 0 & 0 & 0 & 0 \\
0 & 0 & 0 & 0 & 1 & 0 & 0 & 0 \\
0 & 0 & 0 & 0 & 0 & 1 & 0 & 0 \\
0 & 0 & 0 & 0 & 0 & 0 & 0 & 1 \\
0 & 0 & 0 & 0 & 0 & 0 & 1 & 0
\end{bmatrix}$
\normalsize \\
\bottomrule
\end{tabular}
\end{tabularx}
\end{table}




\begin{figure}[h]
\centering
\begin{quantikz}
\lstick{$\ket{0}$} & \gate{H} & \ctrl{1} & \meter{} \\
\lstick{$\ket{0}$} & \qw    & \targ{}  & \meter{}
\end{quantikz}
\caption{Quantum circuit for preparing and measuring a Bell state}
\end{figure}



\end{document}
